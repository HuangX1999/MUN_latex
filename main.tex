%%%%%%%%%%%%%%%%%%%%%%%%%
%模拟联合国决议草案Latex模版
%模版作者:黄鑫,有关于本Latex模版
%的技术性问题或改进意见请通过邮箱
%2357887h@studen.gla.ac.uk与我取得联系
%Date: 2020/04/29
%%%%%%%%%%%%%%%%%%%%%%%%%

%对Latex基本操作不了解的同学
%请不要随意调整无注释部分的内容!

\documentclass[12pt,a4paper]{report}
%调整纸张大小与字号
\usepackage[top=1in, bottom=1in, left=1.3in,right=1.3in]{geometry} 
%调整页边距
\usepackage{color}
\definecolor{dkgreen}{rgb}{0,0.6,0}
\definecolor{gray}{rgb}{0.5,0.5,0.5}
\definecolor{mauve}{rgb}{0.58,0,0.82}
%调整颜色
\usepackage{graphicx} 
\usepackage[dvips, bookmarks, colorlinks=false]{hyperref}
\usepackage{float}
\usepackage{pslatex} 
\usepackage{setspace}
\usepackage{array} 
\usepackage{fancyhdr}
\fancypagestyle{plain}{%
\fancyfoot[L]{\emph{Model United Nations - Documents}}
\fancyfoot[R]{\thepage}
\renewcommand{\headrulewidth}{0.4pt}
\renewcommand{\footrulewidth}{0.4pt}
}
\pagestyle{fancy}
\renewcommand{\chaptermark}[1]{% 
\markboth{#1}{}} 
\fancyfoot[LO,LE]{\emph{Draft Resolution - Version 1.1}}
%调整文章底部标注,如:工作文件1.1,决议草案1.1
\cfoot{}
\fancyfoot[RO, RE]{\thepage}
\renewcommand{\headrulewidth}{0.4pt}
\renewcommand{\footrulewidth}{0.4pt}

%正文部分
\begin{document}

\begin{flushright}
\includegraphics[scale=0.3]{mun-logo}\\
\end{flushright}
\newline
\setlength{\baselineskip}{16pt}
%调整行间距
\textbf{Committee:}
Economic and Fiancial Committee
%输入委员会名称
\newline
\textbf{Sponsor:}
%输入起草国
China, USA, UK
\newline
\textbf{Signatory:}
%输入附议国
Japan
\newline
%在header.tex中编辑委员会信息、起草国以及附议国
\setlength{\baselineskip}{15pt}
%调整行间距
\newline
The General Assembly, %委员会名称
\\%换行

Adopts the following outcome document of the United Nations summit for the adoption of the post-2015 development agenda:
\begin{center} %居中处理
\textbf{Transforming our world: the 2030 Agenda for Sustainable Development}
\end{center}
\newline
\textbf{Preamble}~{}\\

This Agenda is a plan of action for people, planet and prosperity. It also seeks to strengthen universal peace in larger freedom. We recognize that eradicating poverty in all its forms and dimensions, including extreme poverty, is the greatest global challenge and an indispensable requirement for sustainable development.\\

All countries and all stakeholders, acting in collaborative partnership, will implement this plan. We are resolved to free the human race from the tyranny of poverty and want and to heal and secure our planet. We are determined to take the bold and transformative steps which are urgently needed to shift the world on to a sustainable and resilient path. As we embark on this collective journey, we pledge that no one will be left behind.\\

The 17 Sustainable Development Goals and 169 targets which we are announcing today demonstrate the scale and ambition of this new universal Agenda. They seek to build on the Millennium Development Goals and complete what they did not achieve. They seek to realize the human rights of all and to achieve gender equality and the empowerment of all women and girls. They are integrated and indivisible and balance the three dimensions of sustainable development: the economic, social and environmental.\\

The Goals and targets will stimulate action over the next 15 years in areas of critical importance for humanity and the planet.

%在body1.tex中编辑正文信息(序言性条款、程序性条款)
\end{document}
